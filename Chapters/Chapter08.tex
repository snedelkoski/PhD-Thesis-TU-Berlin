% Chapter X

\chapter{Conclusion} % Chapter title

\label{ch:conclusion} % For referencing the chapter elsewhere, use \autoref{ch:name} 
The thesis presented methods for anomaly detection to improve the development, operation, and reliability of distributed software systems. 

For metric data, we presented an unsupervised anomaly detection method based on a variational autoencoder with an RNN as the encoder and decoder to capture both stochastic and sequential properties. In addition to the model, we described a dynamic error threshold approach and tolerance module for false positive reduction. The detected anomaly patterns were then enriched with a corresponding pattern description. We demonstrated the efficiency of the method for both experimental and real-world production data, where we reached an average F1 score of 0.85, prediction time smaller than 10 ms, and robust classification of detected anomalies.

The thesis further contributes to the analysis of log data in log parsing and log anomaly detection. We presented a novel parsing technique, NuLog, which utilizes a self-supervised learning model and formulates the parsing task as MLM. The parsing performance of NuLog evaluated on 10 real-world log datasets outperformed those of existing methods in PA reaching an average PA of 99\% and smallest edit distance to the ground truth templates. Furthermore, we demonstrated the ability of NuLog to extract summarizations from the logs in the form of log vectors. This enabled coupling of the model with a downstream anomaly detection task. We showcased two downstream tasks for log-based anomaly detection in supervised and unsupervised learning scenarios. This showed a large gap between the supervised and unsupervised scenarios. We identified that learning expressive log vector representation is a major factor that helps bridge the gap. In this regard, we presented a new anomaly detection approach, Logsy, a self-attention encoder network with a new hyperspherical classification objective. It learns log vectors and predicts anomaly scores in an end-to-end learning. We elaborated the core principle of the method, aiming to represent normal log samples with small distances in-between (close to the center of a hypersphere), while mapping easily accessible anomalies in the distant space. Due to a large amount of easily obtainable log data, it is reasonable to assume that access to  anomalous samples is available. The efficiency of the method was demonstrated on benchmark datasets where it outperformed the baselines by an F1 score of 0.25. Furthermore, we evaluated several properties of Logsy including utilization of expert knowledge input, effect of the auxiliary data, and transfer of the learned log vector representations in other methods, e.g., in PCA, where we obtained an improvement in the F1 score of 0.07 (28.2\%). We show that incorporating richer domain bias to emphasise the diversity of normal and anomaly data, such as the inclusion of auxiliary data, proves to be beneficial for the anomaly detection.

To cover the request-centric information in distributed systems that contain information on numerous components, we addressed the anomaly detection from the tracing data. We presented Tracy, which aims to model the normal traces by learning to predict a masked span on a particular position in the trace utilizing the remaining nonmasked information from the trace. The decision for the normality of a trace is carried out with a threshold procedure on top of the masked event prediction procedure. Through the evaluations, we demonstrated that the method outperforms the state of the art on experimental testbed data and achieves high scores on data from the global cloud provider. The self-attention mechanism and MSP learning task enable the method to not depend on the length of the trace. This reduced the effect of the noise and improved the generalization of the method.

Lastly, we analyzed complex anomalies in distributed software systems. Through several empirical studies, we demonstrated that the combination of the presented methods broadens the spectrum of detected anomalies and improves the anomaly detection compared to the single methods. 

Although this thesis already addressed central aspects of anomaly detection in distributed software systems, several directions for further investigation exist. These directions can be derived
from the key limitations of the current methods. In metric anomaly detection, a possible extension is to adopt Transformers~\cite{vaswani2017attention} instead of RNNs, which could further improve the effectiveness, particularly by capturing long-term patterns. In log anomaly detection, a possible extension to the anomaly detection method, which operates per log message, is to correlate the detected anomalies on the temporal axis. This may help mine important sequences and processes in the systems, relate them to the trace data, create situation reports, and provide even a larger reduction in the number of FPs. In distributed traces, learning better span representations by direct inclusion of the full span's information may improve the effectiveness of the presented method. Regarding the combination of all detectors and data sources, a possible extension is to integrate the metrics, logs, and traces on a logical level with the use of domain knowledge and system topology. This still remains a challenge that if addressed may enable better modeling and anomaly detection. Overall, the thesis presented a set of methods, which support the development of fully autonomous systems.